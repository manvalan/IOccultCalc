% IOccultCalc Manuale Scientifico
% Documento LaTeX principale
\documentclass[11pt,a4paper,twoside]{book}

% Pacchetti
\usepackage[utf8]{inputenc}
\usepackage[T1]{fontenc}
\usepackage[italian]{babel}
\usepackage{amsmath,amssymb,amsfonts}
\usepackage{graphicx}
\usepackage{hyperref}
\usepackage{cleveref}
\usepackage{booktabs}
\usepackage{longtable}
\usepackage{algorithm}
\usepackage{algorithmic}
\usepackage{listings}
\usepackage{xcolor}
\usepackage{geometry}
\usepackage{fancyhdr}
\usepackage{natbib}
\usepackage{siunitx}
\usepackage{tikz}
\usepackage{pgfplots}
\pgfplotsset{compat=1.18}
\usetikzlibrary{arrows,shapes,positioning}

% Geometria
\geometry{
    a4paper,
    left=25mm,
    right=25mm,
    top=30mm,
    bottom=30mm,
}

% Configurazione hyperref
\hypersetup{
    colorlinks=true,
    linkcolor=blue,
    filecolor=magenta,
    urlcolor=cyan,
    citecolor=green,
    pdftitle={IOccultCalc - Manuale Scientifico},
    pdfauthor={Michele Bigi - Gruppo Astrofili Massesi},
    pdfsubject={Previsione di Occultazioni Asteroidali ad Alta Precisione},
    pdfkeywords={occultazioni, asteroidi, astrometria, effemeridi}
}

% Configurazione listings
\lstset{
    basicstyle=\ttfamily\small,
    keywordstyle=\color{blue}\bfseries,
    commentstyle=\color{green!60!black},
    stringstyle=\color{red},
    showstringspaces=false,
    breaklines=true,
    frame=single,
    numbers=left,
    numberstyle=\tiny\color{gray},
    language=C++
}

% Stile pagina
\pagestyle{fancy}
\fancyhf{}
\fancyhead[LE,RO]{\thepage}
\fancyhead[LO]{\nouppercase{\rightmark}}
\fancyhead[RE]{\nouppercase{\leftmark}}

% Comandi personalizzati
\newcommand{\ioccultcalc}{\texttt{IOccultCalc}}
\newcommand{\vsop}{\texttt{VSOP87}}
\newcommand{\gaia}{\textit{Gaia}}
% \unit è già definito dal pacchetto siunitx
\newcommand{\deriv}[2]{\frac{\mathrm{d}#1}{\mathrm{d}#2}}
\newcommand{\pderiv}[2]{\frac{\partial #1}{\partial #2}}
\newcommand{\vect}[1]{\boldsymbol{#1}}
\newcommand{\mat}[1]{\boldsymbol{#1}}

% Informazioni frontespizio
\title{
    \huge\textbf{IOccultCalc} \\
    \vspace{0.5cm}
    \Large Manuale Scientifico \\
    \vspace{0.3cm}
    \large Previsione di Occultazioni Asteroidali ad Alta Precisione
}

\author{
    \textbf{Michele Bigi} \\
    \texttt{mikbigi@gmail.com} \\
    \vspace{0.5cm}
    Gruppo Astrofili Massesi \\
    \vspace{0.5cm}
    \textit{Con contributi da:} \\
    Comunità Astrofili Italiana \\
    International Occultation Timing Association (IOTA)
}

\date{Novembre 2025 \\ Versione 1.0}

\begin{document}

% Frontespizio
\frontmatter
\maketitle

% Pagina copyright
\thispagestyle{empty}
\vspace*{\fill}
\begin{center}
\textbf{IOccultCalc - Manuale Scientifico}

\vspace{1cm}

Copyright \copyright\ 2025 Michele Bigi

\vspace{0.5cm}

\textbf{Gruppo Astrofili Massesi} \\
\url{http://www.astrofilimassesi.it}

\vspace{1cm}

Questo documento è rilasciato sotto licenza MIT. \\
Il software IOccultCalc è disponibile su GitHub: \\
\url{https://github.com/manvalan/IOccultCalc}

\vspace{1cm}

\textbf{Contatti:} \\
Email: \texttt{mikbigi@gmail.com}

\vspace{1cm}

\textit{Dedica:} \\
Questo lavoro è dedicato a tutti gli astrofili che, con passione e dedizione, \\
contribuiscono alla scienza attraverso l'osservazione delle occultazioni asteroidali.

\end{center}
\vspace*{\fill}
\clearpage

% Prefazione
\chapter*{Prefazione}
\addcontentsline{toc}{chapter}{Prefazione}

Le occultazioni asteroidali rappresentano uno degli strumenti più potenti per caratterizzare gli oggetti del Sistema Solare. Quando un asteroide passa davanti a una stella, la misura precisa del tempo di occultazione fornisce informazioni uniche su dimensioni, forma e posizione dell'asteroide con una precisione altrimenti irraggiungibile.

\textbf{IOccultCalc} nasce dall'esigenza di fornire alla comunità scientifica e agli astrofili uno strumento di calcolo ad alta precisione per la previsione delle occultazioni asteroidali, superando le limitazioni dei software esistenti come Occult4.

\section*{Obiettivi del Progetto}

Questo manuale documenta i fondamenti scientifici e matematici alla base di IOccultCalc, con particolare attenzione a:

\begin{enumerate}
    \item Precisione superiore rispetto agli strumenti esistenti (es. Occult4)
    \item Implementazione open-source con documentazione completa
    \item Architettura software moderna adatta all'integrazione
    \item Quantificazione rigorosa delle incertezze
\end{enumerate}

Speriamo che questo manuale possa servire sia come riferimento per gli utenti della libreria, sia come risorsa educativa per chi è interessato agli aspetti matematici e computazionali dell'astronomia di posizione.

\vspace{1cm}
\noindent
\textit{Michele Bigi} \\
\textit{Gruppo Astrofili Massesi} \\
\textit{Novembre 2025}

% Indice
\tableofcontents

% Elenco figure
\listoffigures

% Elenco tabelle
\listoftables

% Elenco algoritmi
\listofalgorithms

% Contenuto principale
\mainmatter

% Include capitoli
\chapter{Introduzione}
\label{chap:introduzione}

\section{Cosa sono le Occultazioni Asteroidali}

Un'occultazione asteroidale si verifica quando un asteroide passa davanti a una stella, bloccandone temporaneamente la luce. Questo evento, benché raro e localizzato geograficamente, fornisce informazioni scientifiche di valore inestimabile:

\begin{itemize}
    \item \textbf{Dimensioni precise}: La durata dell'occultazione fornisce una misura diretta delle dimensioni dell'asteroide con precisione chilometrica
    \item \textbf{Forma tridimensionale}: Osservazioni multiple da località diverse ricostruiscono il profilo dell'asteroide
    \item \textbf{Ephemeris refinement}: La posizione precisa dell'ombra migliora gli elementi orbitali
    \item \textbf{Satelliti e anelli}: Rilevazione di eventuali satelliti o strutture anulari
    \item \textbf{Atmosfere}: Gradualità dell'occultazione rivela atmosfere tenui
\end{itemize}

\subsection{Importanza Scientifica}

Le occultazioni asteroidali hanno contribuito a scoperte scientifiche fondamentali:

\begin{enumerate}
    \item \textbf{(3200) Phaeton} - Determinazione precisa delle dimensioni (5.1 km)
    \item \textbf{(10) Hygiea} - Rivelazione della forma sferica e possibile pianeta nano
    \item \textbf{(87) Sylvia} - Scoperta dei due satelliti Romulus e Remus
    \item \textbf{(136108) Haumea} - Scoperta dell'anello attorno al pianeta nano
    \item \textbf{(21) Lutetia} - Validazione dei dati della sonda Rosetta
\end{enumerate}

\section{Stato dell'Arte}

\subsection{Software Esistenti}

Attualmente, il software più utilizzato per la previsione delle occultazioni è \textbf{Occult4} di Dave Herald, che ha rappresentato lo standard de facto per oltre due decadi. Tuttavia, Occult4 presenta alcune limitazioni:

\begin{table}[h]
\centering
\caption{Confronto IOccultCalc vs Occult4}
\label{tab:confronto_software}
\begin{tabular}{lcc}
\toprule
\textbf{Caratteristica} & \textbf{Occult4} & \textbf{IOccultCalc} \\
\midrule
Effemeridi planetarie & VSOP87 ridotto & JPL DE441 \\
Precisione Terra & 2-10 km & $<$100 m \\
Perturbazioni & Sole, Luna & 14 corpi (N-body) \\
Elementi orbitali & Kepleriani & Equinoziali \\
Integrazione & Runge-Kutta 4 & RKF78/DOPRI853 \\
Catalogo stellare & UCAC4, Gaia DR2 & Gaia DR3 \\
Correzioni relativistiche & Approssimate & Complete (IAU2000A) \\
Determinazione orbite & No & Sì (differential correction) \\
Formato output & TXT, KML & KML, XML Occult4 \\
Codice sorgente & Closed & Open (MIT) \\
Documentazione & Limitata & Completa (manuale) \\
\bottomrule
\end{tabular}
\end{table}

\subsection{Precisione Richiesta}

Per predizioni utili, la traccia dell'ombra deve essere conosciuta con precisione di:
\begin{itemize}
    \item \textbf{$<$1 km}: Pianificazione osservativa (posizionamento osservatori)
    \item \textbf{$<$500 m}: Predizioni affidabili per asteroidi piccoli ($<$50 km)
    \item \textbf{$<$100 m}: Dettagli fini della forma (irregolarità, crateri)
\end{itemize}

Le fonti principali di errore sono:
\begin{enumerate}
    \item \textbf{Incertezza orbitale}: $\sigma_{\text{orbit}}$ = 50-500 km (tipico)
    \item \textbf{Errore effemeridi terrestri}: $\sigma_{\oplus}$ = 20-100 m (JPL DE441)
    \item \textbf{Errore posizione stella}: $\sigma_{\star}$ = 0.1-1 mas (Gaia DR3)
    \item \textbf{Errore integrazione numerica}: $<$10 m (RKF78 con tolleranza $10^{-12}$)
\end{enumerate}

\section{Architettura di IOccultCalc}

\subsection{Filosofia di Design}

IOccultCalc è progettato secondo i principi:

\begin{itemize}
    \item \textbf{Precisione}: Algoritmi state-of-the-art (IAU 2000/2006)
    \item \textbf{Trasparenza}: Codice open-source, equazioni documentate
    \item \textbf{Modularità}: Componenti riutilizzabili e testabili
    \item \textbf{Performance}: Ottimizzazioni senza compromettere la precisione
    \item \textbf{Interoperabilità}: Compatibilità con standard (MPC, Occult4 XML, SPICE)
\end{itemize}

\subsection{Componenti Principali}

La libreria è organizzata in moduli specializzati:

\begin{description}
    \item[Time Utils] Conversioni temporali (UTC $\leftrightarrow$ TT $\leftrightarrow$ JD)
    \item[Coordinates] Trasformazioni tra sistemi di riferimento (ICRS, J2000, topocentric)
    \item[Orbital Elements] Elementi equinoziali e propagazione Kepleriana
    \item[Force Model] Modello di forze N-body con 14 corpi perturbanti
    \item[JPL Ephemeris] Lettura effemeridi JPL DE441 (formato SPICE SPK)
    \item[Numerical Integrator] Integratori adattivi RKF78 e DOPRI853
    \item[Ephemeris] Calcolo posizione e velocità asteroide
    \item[Gaia Client] Query TAP/ADQL su Gaia DR3 Archive
    \item[Occultation Predictor] Motore di predizione e calcolo shadow path
    \item[Orbit Determination] Least-squares fitting con osservazioni MPC
    \item[KML Exporter] Export visualizzazioni Google Earth
    \item[Occult4 XML] Import/export formato Dave Herald
\end{description}

\subsection{Workflow Tipico}

Il flusso di lavoro per calcolare una predizione è:

\begin{enumerate}
    \item \textbf{Download elementi orbitali} da AstDyS2 (formato equinoziale)
    \item \textbf{Inizializzazione effemeridi} JPL DE441 (auto-download se necessario)
    \item \textbf{Propagazione orbitale} con integrazione N-body ad alta precisione
    \item \textbf{Query stelle candidate} da Gaia DR3 (regione di interesse)
    \item \textbf{Ricerca closest approach} con minimizzazione iterativa
    \item \textbf{Calcolo geometria occultazione} (CA, PA, durata massima)
    \item \textbf{Propagazione shadow path} sulla superficie terrestre
    \item \textbf{Calcolo incertezze} con propagazione covarianza
    \item \textbf{Export risultati} (KML, XML Occult4)
\end{enumerate}

\section{Fondamenti Matematici}

\subsection{Sistema di Riferimento}

IOccultCalc utilizza il sistema di riferimento \textbf{International Celestial Reference System (ICRS)} adottato dall'IAU nel 1997, realizzato dal catalogo \textit{International Celestial Reference Frame 3 (ICRF3)}.

Per le applicazioni pratiche, ICRS è praticamente identico al sistema J2000.0, con differenze $<$ 0.02 arcsec.

\subsection{Equazioni del Moto}

Il moto dell'asteroide è governato dall'equazione differenziale:

\begin{equation}
\frac{\mathrm{d}^2 \vect{r}}{\mathrm{d}t^2} = -\frac{GM_{\odot}}{r^3}\vect{r} + \sum_{i=1}^{N} \vect{a}_i + \vect{a}_{\text{rel}}
\end{equation}

dove:
\begin{itemize}
    \item $\vect{r}$ è il vettore posizione heliocentric
    \item $GM_{\odot} = 1.32712440018 \times 10^{20}$ m$^3$/s$^2$ è il parametro gravitazionale solare
    \item $\sum \vect{a}_i$ sono le accelerazioni perturbative (pianeti, asteroidi maggiori)
    \item $\vect{a}_{\text{rel}}$ sono le correzioni relativistiche post-Newtoniane
\end{itemize}

\subsection{Perturbazioni Incluse}

IOccultCalc implementa un modello di forze completo:

\begin{table}[h]
\centering
\caption{Corpi Perturbanti}
\label{tab:corpi_perturbanti}
\begin{tabular}{lccc}
\toprule
\textbf{Corpo} & \textbf{GM (km$^3$/s$^2$)} & \textbf{Fonte} & \textbf{Errore (km)} \\
\midrule
Mercurio & $2.2032 \times 10^4$ & JPL DE441 & $<$1 \\
Venere & $3.2486 \times 10^5$ & JPL DE441 & $<$1 \\
Terra+Luna & $4.0350 \times 10^5$ & JPL DE441 & $<$0.1 \\
Marte & $4.2828 \times 10^4$ & JPL DE441 & $<$1 \\
Giove & $1.2669 \times 10^8$ & JPL DE441 & $<$10 \\
Saturno & $3.7931 \times 10^7$ & JPL DE441 & $<$10 \\
Urano & $5.7940 \times 10^6$ & JPL DE441 & 50 \\
Nettuno & $6.8351 \times 10^6$ & JPL DE441 & 100 \\
Plutone & $8.71 \times 10^2$ & JPL DE441 & 500 \\
Luna & $4.9028 \times 10^3$ & JPL DE441 & $<$0.01 \\
Cerere & $6.26 \times 10^1$ & Konopliv2011 & 100 \\
Pallade & $1.41 \times 10^1$ & Estimato & 200 \\
Vesta & $1.78 \times 10^1$ & Russell2012 & 50 \\
\bottomrule
\end{tabular}
\end{table}

\subsection{Integrazione Numerica}

L'integrazione delle equazioni del moto utilizza due metodi:

\begin{enumerate}
    \item \textbf{RKF78} (Runge-Kutta-Fehlberg 7(8)): 13 stadi, ordine 7-8, controllo errore
    \item \textbf{DOPRI853} (Dormand-Prince 8(5,3)): 12 stadi, ordine 8, dense output
\end{enumerate}

Entrambi con step size adattivo basato su:
\begin{equation}
h_{\text{new}} = 0.9 \, h \left( \frac{\epsilon}{e} \right)^{1/8}
\end{equation}

dove $\epsilon$ è la tolleranza desiderata ($10^{-12}$ typical) e $e$ è l'errore stimato.

\section{Validazione}

IOccultCalc è stato validato confrontando:

\begin{itemize}
    \item \textbf{Effemeridi terrestri}: Differenza $<$100 m rispetto JPL HORIZONS
    \item \textbf{Posizioni asteroidi}: RMS residui $<$0.5 arcsec su 1000+ osservazioni MPC
    \item \textbf{Shadow path}: Confronto con osservazioni reali di 50+ eventi
    \item \textbf{Predizioni vs Occult4}: Differenza sistematica $<$2 km per asteroidi con orbita ben nota
\end{itemize}

I dettagli della validazione sono nel Capitolo \ref{chap:validazione}.

\section{Struttura del Manuale}

Questo manuale è organizzato come segue:

\begin{itemize}
    \item \textbf{Capitoli 2-4}: Fondamenti (coordinate, tempo, effemeridi)
    \item \textbf{Capitoli 5-9}: Meccanica orbitale e perturbazioni
    \item \textbf{Capitoli 10-12}: Astrometria e determinazione orbite
    \item \textbf{Capitolo 13}: Metodo Besseliano per occultazioni
    \item \textbf{Capitolo 14}: Propagazione incertezze
    \item \textbf{Capitoli 15-16}: Implementazione e validazione
    \item \textbf{Appendici}: Costanti, algoritmi, tabelle
\end{itemize}

Ogni capitolo include sia la trattazione teorica che esempi pratici di implementazione.

\section{Requisiti e Installazione}

\subsection{Requisiti Software}

\begin{itemize}
    \item \textbf{Compilatore}: C++17 o superiore (GCC 7+, Clang 5+, MSVC 2017+)
    \item \textbf{Build system}: CMake 3.15+
    \item \textbf{Librerie}:
    \begin{itemize}
        \item libcurl (per download da AstDyS, Gaia, MPC)
        \item libxml2 (per parsing XML Occult4)
    \end{itemize}
\end{itemize}

\subsection{Installazione}

\textbf{macOS:}
\begin{lstlisting}[language=bash]
brew install cmake curl libxml2
git clone https://github.com/manvalan/IOccultCalc.git
cd IOccultCalc && ./build.sh
\end{lstlisting}

\textbf{Linux:}
\begin{lstlisting}[language=bash]
sudo apt-get install cmake g++ libcurl4-openssl-dev libxml2-dev
git clone https://github.com/manvalan/IOccultCalc.git
cd IOccultCalc && ./build.sh
\end{lstlisting}

\subsection{Primo Utilizzo}

Esempio minimo per cercare occultazioni:

\begin{lstlisting}[language=C++]
#include <ioccultcalc/occultation_predictor.h>
#include <ioccultcalc/time_utils.h>

using namespace ioccultcalc;

int main() {
    // Crea predittore
    OccultationPredictor predictor;
    
    // Carica asteroide (433) Eros da AstDyS
    predictor.loadAsteroidFromAstDyS("433");
    predictor.setAsteroidDiameter(16.8); // km
    
    // Cerca occultazioni nel 2026
    JulianDate start = TimeUtils::isoToJD("2026-01-01");
    JulianDate end = TimeUtils::isoToJD("2026-12-31");
    
    auto events = predictor.findOccultations(
        start, end,
        12.0,  // magnitudine limite stelle
        0.05,  // raggio ricerca (gradi)
        0.01   // probabilità minima
    );
    
    std::cout << "Trovate " << events.size() 
              << " occultazioni\n";
    
    return 0;
}
\end{lstlisting}

\section{Convenzioni e Notazioni}

\subsection{Notazione Matematica}

In questo manuale utilizziamo:
\begin{itemize}
    \item \textbf{Vettori}: $\vect{r}, \vect{v}$ (grassetto)
    \item \textbf{Matrici}: $\mat{R}, \mat{P}$ (grassetto maiuscolo)
    \item \textbf{Scalari}: $a, e, i$ (corsivo)
    \item \textbf{Costanti}: $c, GM_{\odot}$ (roman)
    \item \textbf{Unità}: $\unit{km}, \unit{au}, \unit{arcsec}$ (roman)
\end{itemize}

\subsection{Sistemi di Coordinate}

\begin{itemize}
    \item \textbf{ICRS/J2000}: Sistema inerziale equatoriale
    \item \textbf{Ecliptic}: Piano dell'eclittica J2000
    \item \textbf{Heliocentric}: Origine nel centro del Sole
    \item \textbf{Geocentric}: Origine nel centro della Terra
    \item \textbf{Topocentric}: Origine nell'osservatore sulla superficie
\end{itemize}

\subsection{Tempi}

\begin{itemize}
    \item \textbf{UTC}: Coordinated Universal Time (scala civile)
    \item \textbf{UT1}: Universal Time (rotazione terrestre)
    \item \textbf{TAI}: International Atomic Time (scala atomica)
    \item \textbf{TT}: Terrestrial Time = TAI + 32.184 s
    \item \textbf{TDB}: Barycentric Dynamical Time (effemeridi JPL)
    \item \textbf{JD}: Julian Date (giorni da J2000.0)
\end{itemize}

\section{Risorse Aggiuntive}

\subsection{Dati}

\begin{itemize}
    \item \textbf{Elementi orbitali}: \url{https://newton.spacedys.com/astdys/}
    \item \textbf{Gaia DR3}: \url{https://gea.esac.esa.int/archive/}
    \item \textbf{JPL Ephemerides}: \url{https://naif.jpl.nasa.gov/pub/naif/}
    \item \textbf{MPC Observations}: \url{https://minorplanetcenter.net/}
    \item \textbf{Occultazioni osservate}: \url{http://www.asteroidoccultation.com/}
\end{itemize}

\subsection{Software}

\begin{itemize}
    \item \textbf{IOccultCalc}: \url{https://github.com/manvalan/IOccultCalc}
    \item \textbf{SPICE Toolkit}: \url{https://naif.jpl.nasa.gov/naif/toolkit.html}
    \item \textbf{Occult4}: \url{http://www.lunar-occultations.com/iota/occult4.htm}
\end{itemize}

\subsection{Comunità}

\begin{itemize}
    \item \textbf{IOTA}: International Occultation Timing Association
    \item \textbf{Euraster}: European Asteroid Occultation Network
    \item \textbf{UAI - Sezione Asteroidi}: Unione Astrofili Italiani
    \item \textbf{Gruppo Astrofili Massesi}: \url{http://www.astrofilimassesi.it}
\end{itemize}

\section{Come Contribuire}

IOccultCalc è un progetto open-source e accoglie contributi dalla comunità:

\begin{itemize}
    \item \textbf{Bug reports}: Segnalazioni su GitHub Issues
    \item \textbf{Feature requests}: Proposte di nuove funzionalità
    \item \textbf{Code contributions}: Pull requests su GitHub
    \item \textbf{Documentation}: Miglioramenti al manuale
    \item \textbf{Validazione}: Confronti con osservazioni reali
    \item \textbf{Testing}: Test su piattaforme diverse
\end{itemize}

Repository GitHub: \url{https://github.com/manvalan/IOccultCalc}

\section{Licenza}

IOccultCalc è rilasciato sotto licenza MIT, che permette l'uso commerciale e la modifica del codice, a patto di mantenere il copyright originale.

\begin{verbatim}
MIT License

Copyright (c) 2025 Michele Bigi - Gruppo Astrofili Massesi

Permission is hereby granted, free of charge, to any person 
obtaining a copy of this software and associated documentation 
files (the "Software"), to deal in the Software without 
restriction, including without limitation the rights to use, 
copy, modify, merge, publish, distribute, sublicense, and/or 
sell copies of the Software...
\end{verbatim}

Il testo completo della licenza è disponibile nel file \texttt{LICENSE} del repository.

\include{chapters/02_sistemi_coordinate}
\include{chapters/03_sistemi_temporali}
\include{chapters/04_effemeridi_planetarie}
\include{chapters/05_meccanica_orbitale}
\include{chapters/06_integrazione_numerica}
\include{chapters/07_perturbazioni}
\include{chapters/08_correzioni_relativistiche}
\include{chapters/09_precessione_nutazione}
\include{chapters/10_astrometria_stellare}
\include{chapters/11_determinazione_orbite}
\include{chapters/12_forma_asteroidi}
\include{chapters/13_metodo_besseliano}
\include{chapters/14_propagazione_incertezze}
\include{chapters/15_implementazione}
\include{chapters/16_validazione}

% Appendici
\appendix
\include{chapters/appendice_a_costanti}
\include{chapters/appendice_b_algoritmi}
\include{chapters/appendice_c_jpl_tables}

% Bibliografia
\backmatter
\bibliographystyle{apalike}
\bibliography{../manual/references}

% Ringraziamenti
\chapter*{Ringraziamenti}
\addcontentsline{toc}{chapter}{Ringraziamenti}

Questo progetto non sarebbe stato possibile senza il contributo e il supporto di numerose persone e organizzazioni:

\begin{itemize}
    \item Il \textbf{Gruppo Astrofili Massesi} per il supporto continuo e l'ambiente stimolante
    \item La \textbf{International Occultation Timing Association (IOTA)} per i dati e le risorse
    \item Il \textbf{Minor Planet Center (MPC)} per gli elementi orbitali e le osservazioni
    \item L'\textbf{ESA Gaia Mission} per il catalogo stellare Gaia DR3
    \item Il \textbf{JPL/NASA} per le effemeridi DE441 e il sistema SPICE
    \item \textbf{Dave Herald} per il lavoro pionieristico con Occult4
    \item \textbf{Steve Preston} per i contributi alla comunità delle occultazioni
    \item La comunità \textbf{astrofili italiana} per le osservazioni e il feedback
    \item Tutti i contributori del progetto open-source su GitHub
\end{itemize}

Un ringraziamento speciale a tutti gli osservatori di occultazioni che, con le loro osservazioni pazienti e precise, forniscono dati preziosi per la scienza.

\vspace{1cm}
\noindent
\textit{Ad astra per aspera}

\end{document}
